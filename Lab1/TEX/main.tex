\documentclass[a4paper,12pt]{extarticle}

\usepackage{geometry}
\geometry{left=1.5cm, right=1.5cm, top=3cm, bottom=3cm}

\usepackage[document]{ragged2e}
%--------------------------------------
\usepackage[T2A]{fontenc}
\usepackage[utf8]{inputenc}
\usepackage[russian]{babel}
%--------------------------------------
\usepackage{enumitem}
\usepackage{fancyhdr}
\pagestyle{fancy}   %to use abilities of package fancyhdr
\renewcommand{\headrulewidth}{0mm}
\fancyhead[C]{Белорусский государственный университет\\информатики и радиоэлектроники\\Кафедра интеллектуальных информационных технологий}
\fancyfoot[C]{Минск 2021}
%--------------------------------------

\begin{document}
%-------------------------------------title page----------------------------------%

\hspace{0pt}
\vfill
\begin{center}
\section*{Лабораторная работа № 2}
\section*{«Операции над множествами»}
\end{center}
\vfill
\begin{tabbing}
Выполнили (cтуденты группы 121703):\\
\hspace{1em} \= Тарбая Данила\\
\hspace{1em} \= Рутковский Александр\\
\hspace{1em} \= Якимович Илья\\
Проверила:\\
\hspace{1em} \= Гулякина Н. А.\\
\end{tabbing}
\hspace{0pt}
\pagebreak

%-------------------------------------text page----------------------------------%

\newpage
\fancyhf{}
\begin{center}
\section*{Постановка задачи}
\end{center}
\justify\ \ \ Даны два множества. Найти их пересечение, объединение, разность, симметричную разность, декартово произведение, дополнение. Множества задаются перечислением или высказыванием.
%---------------------------------------------------------------------------------%
\begin{center}
\section*{Уточнение постановки задачи}
\end{center}

\begin{enumerate}
  \item n - мощность множества А.
  \item m - мощность множества В.
  \item Мощность множества A и B натуральные числа, которые находятся в диапазоне от 0 до 100 и задаются пользователем.
  \item Элементы множества A и B являются натуральными числами в диапазоне от 0 до 100 и задаются пользователем.
  \item Элементами универсального множества U для множества A и B являются натуральные числа на интервале от 1 до 100.
  \item Пользозователь выбирает выполняемую операцию.
  \item Множества могут быть заданы перечислением:

  множества заданные перечислением задаются пользователем

  \item Множества могут быть заданы высказываниями:
  \begin{itemize}
    \item A = $\{a|a \in N, a = \frac{4x(x-2)(x+4)}{4x^2+8x-32}, x = \overline{1,n}\}$
    \item B = $\{b|b \in N, b = \frac{4x(x-2)(x+4)}{4x^2+8x-32}, x = \overline{1,m}\}$
  \end{itemize}
\end{enumerate}
%---------------------------------------------------------------------------------%
\begin{center}
\section*{Используемые понятия}
\end{center}
\begin{itemize}
  \item\textbf{Множество} — это любое собрание определенных и различных между собой объектов нашей интуиции или интеллекта, мыслимое как единое целое. Эти объекты — элементы множества;
  \item\textbf{Мощность множества} — это количество элементов во множестве;
  \item\textbf{Объединение множеств} — это множество, которое состоит из тех элементов, которые принадлежат хотя бы одному из множеств A, B;
  \item\textbf{Пересечение множеств} — это множество, которое состоит из тех элементов, которые  принадлежат множеству A и множеству B одновременно;
  \item\textbf{Разность множеств} - множество, в которое входят все элементы первого множества, не входящие во второе множество.
  \item\textbf{Симметрическая разность} - множество, включающее все элементы исходных множеств, не принадлежащие одновременно обоим исходным множествам.
  \item\textbf{Декартово произведение} - множество, элементами которого являются все возможные упорядоченные пары элементов исходных множеств.
  \item Множество A’ называется \textbf{дополнением} множества A до некоторого универсального множества U, если оно состоит из элементов, принадлежащих множеству U и не принадлежащих множеству A.
  \item \textbf{Кортеж} - упорядоченный набор компонент (элементов).
\end{itemize}
%---------------------------------------------------------------------------------%
\begin{center}
\section*{Алгоритм}
\end{center}
%--------------------------------------main part----------------------------------%
\begin{enumerate}
  \item Пользователь выбирает способ задания множеств
  \begin{enumerate}[label*=\arabic*.]
    \item Если пользователь выбирает способ задания множеств перечислением:
    \begin{enumerate}[label*=\arabic*.]
      \item Переходим к пункту 2.
    \end{enumerate}
    \item Если пользователь выбирает способ задания множество высказыванием:
    \begin{enumerate}[label*=\arabic*.]
      \item Переходим к пункту 3
    \end{enumerate}
  \end{enumerate}
  \item Задание множеств перечислением
  \begin{enumerate}[label*=\arabic*.]
    \item Пользователь задает множество А перечислением:
    \begin{enumerate}[label*=\arabic*.]
      \item Пользователь вводит мощность множества А.
      \item Пользователь вводит элементы множества А.
    \end{enumerate}
    \item Пользователь задает множество В перечислением.
    \begin{enumerate}[label*=\arabic*.]
      \item Пользователь вводит мощность множества В.
      \item Пользователь вводит элементы множества В.
    \end{enumerate}
  \end{enumerate}
  \item Задание множеств высказыванием
  \begin{enumerate}[label*=\arabic*.]
    \item Задаем множества А высказыванием
    \begin{enumerate}[label*=\arabic*.]
      \item Пользователь вводит n — мощность множества А
      \item Присваиваем значение х = 1 (для множества А).
      \item Вычисляем значение a по формуле a = x, подставляя текущее значение x.
      \item Переносим значение a во множество A.
      \item Если значение x больше или равно n, то переходим к пункту 3.1.8
      \item Увеличиваем x на 1.
      \item Переходим к пункту 3.1.3.
      \item А — множество, заданное высказыванием.
      \item Выведем на экран множество А.
    \end{enumerate}
    \item Задаем множество В высказыванием
    \begin{enumerate}[label*=\arabic*.]
      \item Пользователь вводит m — мощность множества В.
      \item Присваиваем значение х = 1 (для множества B).
      \item Вычисляем значение b по формуле b = x, подставляя текущее значение x.
      \item Переносим значение b во множество B.
      \item Если значение x больше или равно m, то переходим к пункту 3.2.8.
      \item Увеличиваем x на 1.
      \item Переходим к пункту 3.2.3.
      \item B — множество, заданное высказыванием.
      \item Выведем на экран множество В.
    \end{enumerate}
  \end{enumerate}
  \newpage
  \item Ввод данных:
  \begin{enumerate}[label*=\arabic*.]
    \item Пользователь задает мощность универсума U
    \item Пользователь задает универсум U
    \item Пользователь задает мощность множества А.
    \item Пользользователь задает множество А.
    \item Пользователь задает мощность множества В.
    \item Пользователь задет множество В.
  \end{enumerate}
  \item Выбор операции:
  \begin{enumerate}[label*=\arabic*.]
    \item Пользователь должен выбрать, какую из операций он хочет выполнить, в зависимости от его выбора будет выполнена операция из следующего списка:
    \begin{itemize}
      \item Объединение.
      \item Пересечение.
      \item Разность.
      \item Симметрическая разность.
      \item Декартово произведение.
      \item Дополнение.
    \end{itemize}
    \item Если пользователь выбрал операцию объединения
    \begin{enumerate}[label*=\arabic*.]
      \item Переходим к пункту 6.
    \end{enumerate}
    \item Если пользователь выбрал операцию пересечения
    \begin{enumerate}[label*=\arabic*.]
      \item Переходим к пункту 7.
    \end{enumerate}
    \item Если пользователь выбрал операцию разности A и В
    \begin{enumerate}[label*=\arabic*.]
      \item Переходим к пункту 8.1
    \end{enumerate}
    \item Если пользователь выбрал операции разности В и А
    \begin{enumerate}[label*=\arabic*.]
      \item Переходим к пункту 8.2
    \end{enumerate}
    \item Если пользователь выбрал операцию симметрической разности
    \begin{enumerate}[label*=\arabic*.]
      \item Переходим к пункту 9.
    \end{enumerate}
    \item Если пользователь выбрал операцию дополнения A
    \begin{enumerate}[label*=\arabic*.]
      \item Переходим к пункту 10.1
    \end{enumerate}
    \item Если пользователь выбрал операцию дополнения В
    \begin{enumerate}[label*=\arabic*.]
      \item Переходим к пункту 10.2
    \end{enumerate}
    \item Если пользователь выбрал операцию Декартова произведения А и В
    \begin{enumerate}[label*=\arabic*.]
      \item Переходим к пункту 11.1
    \end{enumerate}
    \item Если пользователь выбрал операцию Декартова произведения В и А
    \begin{enumerate}[label*=\arabic*.]
      \item Переходим к пункту 11.2
    \end{enumerate}
  \end{enumerate}
  \item \textbf{Операция объединения}:
  \begin{enumerate}[label*=\arabic*.]
    \item Создается пустое множество С, которое будет результатом операции.
    \item Выбираем первый элемент из множества А.
    \item Записываем выбранный элемент из множества А в множество С.
    \item Если выбранный элемент множества А является последним:
    \begin{enumerate}[label*=\arabic*.]
      \item Переходим к пункту 6.6.
    \end{enumerate}
    \item Выбираем следующий элемент множества А
    \begin{enumerate}[label*=\arabic*.]
      \item Переходим к пункту 6.3.
    \end{enumerate}
    \item Выбираем первый элемент множества А.
    \item Выбираем первый элемент множества В.
    \item Сравниваем выбранный элемент из множества А с выбранным элементом из множества В.
    \begin{enumerate}[label*=\arabic*.]
      \item Если выбранный элемент из множества А не равен выбранному элементу из множества В и выбранный элемент из множества А не является последним
      \begin{enumerate}[label*=\arabic*.]
        \item Выбираем следующий элемент множества А
        \begin{enumerate}[label*=\arabic*.]
          \item Переходим к пункту 6.8.
        \end{enumerate}
      \end{enumerate}
      \item Если выбранный элемент из множества А является последним и не равен выбранному элементу из множества В
      \begin{enumerate}[label*=\arabic*.]
        \item Переходим к пункту 6.10.
      \end{enumerate}
      \item Если выбранный элемент из множества А равен выбранному элементу из множества В
      \begin{enumerate}[label*=\arabic*.]
        \item Переходим к пункту 6.9.
      \end{enumerate}
    \end{enumerate}
    \item Рассмотрим следующий элемент из множества В.
    \begin{enumerate}[label*=\arabic*.]
      \item Переходим к пункту 6.8.
    \end{enumerate}
    \item Записываем выбранный элемент из множества В в множество С.
    \begin{enumerate}[label*=\arabic*.]
      \item Если выбранный элемент из множества В является последним
      \begin{enumerate}[label*=\arabic*.]
        \item Переходим к пункту 6.12.
      \end{enumerate}
      \item Выбираем следующий элемент из множества В и первый эелемент из множества А.
    \end{enumerate}
    \item Переходим к пункту 6.8.
    \item Множество С является множеством объединения множеств А и В.
    \item Алгоритм завершен.
  \end{enumerate}
  \item \textbf{Операция пересечения}
  \begin{enumerate}[label*=\arabic*.]
    \item Создаем пустое множество D, которое будет результатом операции.
    \item Выбираем первый элемент множества А.
    \item Выбираем первый элемент множества В.
    \item Если выбранный эелемент множества А равен выбранному элементу множества В, то элемент множества В записывается во множество D.
    \item Если выбранный элемент множества В является последним
    \begin{enumerate}[label*=\arabic*.]
      \item Переходим к пункту 7.7.
    \end{enumerate}
    \item Выбираем следующий элемент множества В.
    \begin{enumerate}[label*=\arabic*.]
      \item Переходим к пункту 7.4.
    \end{enumerate}
    \item Если выбранный элемент множества А является последним.
    \begin{enumerate}[label*=\arabic*.]
      \item Переходим к пункту 7.9.
    \end{enumerate}
    \item Выбираем следующий элемент множества А.
    \begin{enumerate}[label*=\arabic*.]
      \item Переходим к пункту 7.4.
    \end{enumerate}
    \item Множество D является результатом пересечения множеств А и В.
    \item Алгоритм завершен.
  \end{enumerate}
  \item \textbf{Операция разности}
  \begin{enumerate}[label*=\arabic*.]
  \item \textbf{Операция разности A и B}
  \begin{enumerate}[label*=\arabic*.]
    \item Создадим пустое множество D
    \item Возьмём первый элемент множества B.
    \item Возьмём первый элемент множества A.
    \item Если взятый элемент множества B равен взятому элементу множества A
    \begin{enumerate}[label*=\arabic*.]
      \item Переходим к пункту 8.1.9.
    \end{enumerate}
    \item Если взятый элемент множества A является последним
    \begin{enumerate}[label*=\arabic*.]
      \item Переходи к пункту 8.1.8.
    \end{enumerate}
    \item Если взятый элемент множества A не является последним, возьмём следующий элемент множества A.
    \item Перейдём к пункту 8.1.4.
    \item Добавляем взятый элемент множества B в множество D.
    \item Если взятый элемент множества А является последним.
    \begin{enumerate}[label*=\arabic*.]
      \item Перейдём к пункту 12.
    \end{enumerate}
    \item Если взятый элемент множества B не является последним, возьмём следующий элемент множества B.
    \item Перейдём к пункту 8.1.3.
  \end{enumerate}
  \item\textbf{Операция разности B и A}
  \begin{enumerate}[label*=\arabic*.]
    \item Создадим пустое множество D.
    \item Возьмём первый элемент множества B.
    \item Возьмём первый элемент множества A.
    \item Если взятый элемент множества B равен взятому элементу множества A.
    \begin{enumerate}[label*=\arabic*.]
      \item Переходим к пункту 8.2.9.
    \end{enumerate}
    \item Если взятый элемент множества A является последним
    \begin{enumerate}[label*=\arabic*.]
      \item Перейдём к пункту 8.2.8
    \end{enumerate}
    \item Если взятый элемент множества A не является последним, возьмём следующий элемент множества A.
    \item Перейдём к пункту 8.2.4.
    \item Добавляем взятый элемент множества B в множество D.
    \item Если взятый элемент множества А является последним
    \begin{enumerate}[label*=\arabic*.]
      \item Перейдём к пункту 12
    \end{enumerate}
    \item Если взятый элемент множества B не является последним, возьмём следующий элемент множества B.
    \item Перейдём к пункту 8.2.3.
  \end{enumerate}
  \end{enumerate}
  \item \textbf{Симметрическая разность множеств А и В.}
  \begin{enumerate}[label*=\arabic*.]
    \item \textbf{Разность множеств А и В}
    \begin{enumerate}[label*=\arabic*.]
      \item Создадим пустое множество C.
      \item Возьмём первый элемент множества А.
      \item Возьмём первый элемент множества В.
      \item Если взятый элемент множества А равен взятому элементу множества В
      \begin{enumerate}[label*=\arabic*.]
        \item Переходим к пункту 9.1.9.
      \end{enumerate}
      \item Если взятый элемент множества В является последним
      \begin{enumerate}[label*=\arabic*.]
        \item Перейдём к пункту 9.1.8
      \end{enumerate}
      \item Если взятый элемент множества В не является последним, возьмём следующий элемент множества В.
      \item Перейдём к пункту 9.1.4.
      \item Добавляем взятый элемент множества А в множество C.
      \item Если взятый элемент множества А является последним
      \begin{enumerate}[label*=\arabic*.]
        \item Перейдём к пункту 9.1.12.
      \end{enumerate}
      \item Если взятый элемент множества А не является последним, возьмём следующий элемент множества А.
      \item Перейдём к пункту 9.1.3.
      \item C — разность множеств А и В.
      \item Завершение алгоритма
    \end{enumerate}
    \item \textbf{Разность множеств В и А.}
    \begin{enumerate}[label*=\arabic*.]
      \item Создадим пустое множество F.
      \item Возьмём первый элемент множества B.
      \item Возьмём первый элемент множества A.
      \item Если взятый элемент множества B равен взятому элементу множества A
      \begin{enumerate}[label*=\arabic*.]
        \item Переходим к пункту 9.2.9.
      \end{enumerate}
      \item Если взятый элемент множества A является последним
      \begin{enumerate}
        \item Перейдём к пункту 9.2.8
      \end{enumerate}
      \item Если взятый элемент множества A не является последним, возьмём следующий элемент множества A.
      \item Перейдём к пункту 9.2.4.
      \item Добавляем взятый элемент множества B в множество F.
      \item Если взятый элемент множества B является последним
      \begin{enumerate}[label*=\arabic*.]
        \item Перейдём к пункту 9.2.12.
      \end{enumerate}
      \item Если взятый элемент множества B не является последним, возьмём следующий элемент множества B.
      \item Перейдём к пункту 9.2.3.
      \item F — разность множеств B и А.
      \item Завершение алгоритма
    \end{enumerate}
    \item\textbf{Объединение множеств C и F.}
    \begin{enumerate}[label*=\arabic*.]
      \item Создаём новое пустое множество D.
      \item Каждый элемент множества C переносим в множество D.
      \item Возьмём первый элемент множества F.
      \item Возьмём первый элемент множества D.
      \item Если взятый элемент множества F не равен взятому элементу D
      \begin{enumerate}[label*=\arabic*.]
        \item Переходим к пункту 9.3.7.
      \end{enumerate}
      \item Если взятый элемент множества В равен выбранному элементу множества D
      \begin{enumerate}[label*=\arabic*.]
        \item Переходим к пункту 9.3.11
      \end{enumerate}
      \item Если взятый элемент множества D — последний
      \begin{enumerate}[label*=\arabic*.]
        \item Переходим к пункту 9.3.10.
      \end{enumerate}
      \item Если взятый элемент множества D — не последний, то возьмём следующий элемент множества D.
      \item Перейдём к пункту 9.3.5.
      \item Добавляем взятый элемент множества F во множество D.
      \item Если взятый элемент множества F — последний
      \begin{enumerate}[label*=\arabic*.]
        \item Переходим к пункту 12
      \end{enumerate}
      \item Если взятый элемент множества F — не последний, то возьмём следующий элемент множества F.
      \item Перейдём к пункту 9.3.4.
    \end{enumerate}
  \end{enumerate}
  \item \textbf{Операция дополнения}
  \begin{enumerate}[label*=\arabic*.]
    \item \textbf{Дополнение множества А}
    \begin{enumerate}[label*=\arabic*.]
      \item Зададим множество U.
      \begin{enumerate}[label*=\arabic*.]
        \item Присвоим значение x = 1.
        \item Если значение x больше 100, перейдём к пункту 10.1.2.
        \item Добавим значение x во множество U.
        \item x = x + 1.
        \item Перейдём к пункту 10.1.1.2.
      \end{enumerate}
      \item Разность множеств U и A.
      \begin{enumerate}[label*=\arabic*.]
        \item Создадим пустое множество D.
        \item Возьмём первый элемент множества U.
        \item Возьмём первый элемент множества A.
        \item Если взятый элемент множества U равен взятому элементу множества A, то переходим к пункту 10.1.2.11.
        \item Если взятый элемент множества A является последним, то перейдём к пункту 10.1.2.10
        \item Если взятый элемент множества A не является последним, возьмём следующий элемент множества A.
        \item Перейдём к пункту 10.1.2.4.
        \item Добавляем взятый элемент множества U в множество D
        \item Если взятый элемент множества А является последним, перейдём к пункту 12.
        \item Если взятый элемент множества U не является последним, возьмём следующий элемент множества U, то перейдём к пункту 10.1.2.3.
      \end{enumerate}
    \end{enumerate}
    \item \textbf{Дополнение множества B}
    \begin{enumerate}[label*=\arabic*.]
      \item Дополнение множества В
        \begin{enumerate}[label*=\arabic*.]
          \item Присвоим значение x = 1.
          \item Если значение x больше 100, перейдём к пункту 10.2.2
          \item Добавим значение x во множество U.
          \item x = x + 1.
          \item Перейдём к пункту 10.2.1.2
        \end{enumerate}
      \item Разность множеств U и B.
        \begin{enumerate}[label*=\arabic*.]
          \item Создадим пустое множество D.
          \item Возьмём первый элемент множества U.
          \item Возьмём первый элемент множества B.
          \item Если взятый элемент множества U равен взятому элементу множества B, то переходим к пункту 10.2.2.8.
          \item Если взятый элемент множества B является последним, то перейдём к пункту 10.2.2.7.
          \item Если взятый элемент множества B не является последним, то возьмём следующий элемент множества B и перейдём к пункту 10.2.2.4.
          \item Добавляем взятый элемент множества U в множество D.
          \item Если взятый элемент множества B является последним, то перейдём к пункту 12.
          \item Если взятый элемент множества U не является последним, то возьмём следующий элемент множества U и перейдём к пункту 10.2.2.3.
        \end{enumerate}
      \end{enumerate}
    \end{enumerate}
  \item \textbf{Операция Декартова произведения множеств}
  \begin{enumerate}[label*=\arabic*.]
    \item \textbf{Декартово произведение множеств А и В.}
    \begin{enumerate}[label*=\arabic*.]
      \item Создаём пустое множество D.
      \item Возьмём первый элемент множества А.
      \item Возьмём первый элемент множества В.
      \item Создаём кортеж, состоящий из двух элементов:
      \begin{enumerate}[label*=\arabic*.]
        \item Первому элементу кортежа присвоим значение взятого элемента множества А.
        \item Второму элементу кортежа присвоим значение взятого элемента множества В.
      \end{enumerate}
      \item Добавим созданный кортеж во множество D.
      \item Если взятый элемент множества В является последним
      \begin{enumerate}[label*=\arabic*.]
        \item Перейдём к пункту 11.1.9.
      \end{enumerate}
      \item Если взятый элемент множества В не является последним, то возьмём следующий элемент множества В.
      \item Перейдём к пункту 11.1.4.
      \item Если взятый элемент множества А является последним
      \begin{enumerate}[label*=\arabic*.]
        \item Перейдём к пункту 12.
      \end{enumerate}
      \item Если взятый элемент множества А не является последним, то возьмём следующий элемент множества А.
      \item Перейдём к пункту 11.1.3.
    \end{enumerate}
    \item \textbf{Декартово произведение множеств В и А}
    \begin{enumerate}[label*=\arabic*.]
      \item Создаём пустое множество D.
      \item Возьмём первый элемент множества В.
      \item Возьмём первый элемент множества А.
      \item Создаём кортеж, состоящий из двух элементов:
      \begin{enumerate}[label*=\arabic*.]
        \item Первому элементу кортежа присвоим значение взятого элемента множества В.
        \item Второму элементу кортежа присвоим значение взятого элемента множества А.
      \end{enumerate}
      \item Добавим созданный кортеж во множество D.
      \item Если взятый элемент множества А является последним
      \begin{enumerate}[label*=\arabic*.]
        \item Перейдём к пункту 11.2.9
      \end{enumerate}
      \item Если взятый элемент множества А не является последним, то возьмём следующий элемент множества А.
      \item Перейдём к пункту 11.2.4.
      \item Если взятый элемент множества В является последним
      \begin{enumerate}[label*=\arabic*.]
        \item Перейдём к пункту 12.
      \end{enumerate}
      \item Если взятый элемент множества В не является последним, то возьмём следующий элемент множества В.
      \item Перейдём к пункту 11.2.3.
    \end{enumerate}
  \end{enumerate}
  \item Множество D является результатом выполнения выбранной операции.
  \item Завершение алгоритма.
\end{enumerate}





\end{document}
