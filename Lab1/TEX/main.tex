\documentclass[a4paper,12pt]{extarticle}

\usepackage{geometry}
\geometry{left=1.5cm, right=1.5cm, top=3cm, bottom=3cm}

\usepackage[document]{ragged2e}
%--------------------------------------
\usepackage[T2A]{fontenc}
\usepackage[utf8]{inputenc}
\usepackage[russian]{babel}
%--------------------------------------
\usepackage{enumitem}
\usepackage{fancyhdr}
\pagestyle{fancy}   %to use abilities of package fancyhdr
\renewcommand{\headrulewidth}{0mm}
\fancyhead[C]{Белорусский государственный университет\\информатики и радиоэлектроники\\Кафедра интеллектуальных информационных технологий}
\fancyfoot[C]{Минск 2021}
%--------------------------------------

\begin{document}
%-------------------------------------title page----------------------------------%

\hspace{0pt}
\vfill
\begin{center}
\section*{Лабораторная работа № 1}
\section*{«Операции над множествами»}
\end{center}
\vfill
\begin{tabbing}
Выполнили (cтуденты группы 121703):\\
\hspace{1em} \= Торбая Даниил\\
\hspace{1em} \= Рутковский Александр\\
\hspace{1em} \= Якимович Илья\\
Проверила:\\
\hspace{1em} \= Гулякина Н. А.\\
\end{tabbing}
\hspace{0pt}
\pagebreak

%-------------------------------------text page----------------------------------%

\newpage
\fancyhf{}
\begin{center}
\section*{Постановка задачи}
\end{center}
\justify\ \ \ Даны два множества. Найти их пересечение и объединение.
%---------------------------------------------------------------------------------%
\begin{center}
\section*{Уточнение постановки задачи}
\end{center}

\begin{enumerate}
  \item Элементы множества A и B вводятся с клавиатуры.
  \item Элементами множеств A, B являются целые числа.
  \item Мощность множеств A и B находится в диапазоне от 0 до 100 и задается пользователем.
  \item Пользользователь выбирает выполняемую операцию.
\end{enumerate}
%---------------------------------------------------------------------------------%
\begin{center}
\section*{Используемые понятия}
\end{center}
\begin{itemize}
  \item\textbf{Множество} — это любое собрание определенных и различных между собой объектов нашей интуиции или интеллекта, мыслимое как единое целое. Эти объекты — элементы множества;
  \item\textbf{Мощность} множества — это количество элементов во множестве;
  \item\textbf{Объединение множеств} — это множество, которое состоит из тех элементов, которые принадлежат хотя бы одному из множеств A, B;
  \item\textbf{Пересечение множеств A и B} — это множество, которое состоит из тех элементов, которые  принадлежат множеству A и множеству B одновременно;
\end{itemize}
%---------------------------------------------------------------------------------%
\begin{center}
\section*{Алгоритм}
\end{center}
%--------------------------------------main part----------------------------------%
\begin{enumerate}
  \item Ввод данных:
  \begin{enumerate}[label*=\arabic*.]
    \item Пользользователь задает множество А
    \item Пользователь задет множество В
  \end{enumerate}
  \item Выбор операции:
  \begin{enumerate}[label*=\arabic*.]
    \item Пользователь должен выбрать, какую из операций он хочет выполнить, в зависимости от его выбора будет выполнена операция из следующего списка:
    \begin{itemize}
      \item объединение (переходим к пункту 3)
      \item Пересечение (переходим к пункту 4)
    \end{itemize}
  \end{enumerate}
  \item Операция \textbf{объединения}:
  \begin{enumerate}[label*=\arabic*.]
    \item Создается пустое множество С, которое будет результатом Операции
    \item Выбираем первый элемент из множества А
    \item Записываем выбранный элемент из множества А в множество С
    \item Если выбранный элемент множества А является последним, то переходим к пункту 3.6
    \item Выбираем следующий эелемент множества А, переходим к пункту 3.3
    \item Выбираем первый элемент множества А
    \item Выбираем первый эелемент множества В
    \item Проверяем выбранный элемент из множества А с выбранным эелементом из множества В
  \end{enumerate}
\end{enumerate}




















\end{document}
