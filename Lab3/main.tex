\documentclass[a4paper,12pt]{extarticle}
 
\usepackage{geometry}
\geometry{left=1.5cm, right=1.5cm, top=3cm, bottom=3cm}
 
\usepackage[document]{ragged2e}
%--------------------------------------
\usepackage[T2A]{fontenc}
\usepackage[utf8]{inputenc}
\usepackage[russian]{babel}
%--------------------------------------
\usepackage{enumitem}
\usepackage{fancyhdr}
\pagestyle{fancy}   %to use abilities of package fancyhdr
\renewcommand{\headrulewidth}{0mm}
\fancyhead[C]{Белорусский государственный университет\\информатики и радиоэлектроники\\Кафедра интеллектуальных информационных технологий}
\fancyfoot[C]{Минск 2021}
%--------------------------------------
 
\begin{document}
%-------------------------------------title page----------------------------------%
 
\hspace{0pt}
\vfill
\begin{center}
\section*{Лабораторная работа № 3}
\section*{«Операции над графиками»}
\end{center}
\vfill
\begin{tabbing}
Выполнили (cтуденты группы 121703):\\
\hspace{1em} \= Тарбая Данила\\
\hspace{1em} \= Рутковский Александр\\
\hspace{1em} \= Якимович Илья\\
Проверила:\\
\hspace{1em} \= Гулякина Н. А.\\
\end{tabbing}
\hspace{0pt}
\pagebreak
 
%-------------------------------------text page----------------------------------%
 
\newpage
\fancyhf{}
\begin{center}
\section*{Постановка задачи}
\end{center}
\justify\ \ \ Даны два графика. Найти их объединение, пересечение, разность, симметрическую разность, дополнение, композицию, инверсию. Элементы графиков задаются перечислением.
%---------------------------------------------------------------------------------%
\begin{center}
\section*{Уточнение постановки задачи}
\end{center}
 
\begin{enumerate}
  \item Мощность графиков A и B натуральные числа, которые находятся в диапазоне от 1 до 100 и задаются пользователем.
  \item Элементами графика являются упорядоченные пары, элементами которых являются натуральные целые числа в диапазоне от 1 до 100
  \item Элементами универсального множества U для множества A и B являются натуральные числа на интервале от 1 до 100.
  \item За один проход программа выполняет одну операцию, выбранную пользователем.
  \item Элементы обоих графиков вводятся пользователем.
  \item Пользователь сам выбирает, какая операция будет выполняться.
\end{enumerate}
%---------------------------------------------------------------------------------%
\begin{center}
\section*{Используемые понятия}
\end{center}
\begin{itemize}
\tem\textbf{График} - это множество, каждый элемент которого является парой или кортежем длины 2.\\
  \item\textbf{Множество} — это любое собрание определенных и различных между собой объектов нашей интуиции или интеллекта, мыслимое как единое целое. Эти объекты — элементы множества;\\
  \item\textbf{Областью определения графика Р} называется множество пр1P(проекция на первую ось (ось абсцисс) данного графика).\\
  \item\textbf{Областью значений графика P} называется множество проекций на вторую ось (ось ординат) (пр2Р).\\
  \item\textbf{Мощность множества} — это количество элементов во множестве;\\
  \item\textbf{Объединение множеств} — это множество, которое состоит из тех элементов, которые принадлежат хотя бы одному из множеств A, B;\\
  \item\textbf{Пересечение множеств} — это множество, которое состоит из тех элементов, которые  принадлежат множеству A и множеству B одновременно;\\
  \item\textbf{Разность множеств} - множество, в которое входят все элементы первого множества, не входящие во второе множество.\\
  \item\textbf{Симметрическая разность} - множество, включающее все элементы исходных множеств, не принадлежащие одновременно обоим исходным множествам.\\
  \item\textbf{Декартово произведение} - множество, элементами которого являются все возможные упорядоченные пары элементов исходных множеств.\\
  \item \textbf{Дополнением} множества A до некоторого универсального множества U, называется множество A’, если оно состоит из элементов, принадлежащих множеству U и не принадлежащих множеству A.\\
  \item \textbf{Кортеж} - упорядоченный набор компонент (элементов).\\
\end{itemize}
%---------------------------------------------------------------------------------%
\begin{center}
\section*{Алгоритм}
\end{center}
%--------------------------------------main part----------------------------------%
\begin{enumerate}
  \item Ввод данных:
  \begin{enumerate}[label*=\arabic*.]
    \item Пользователь задает мощность универсума U
    \item Пользователь задает универсум U
    \item Пользователь задает мощность графика А.
    \item Пользользователь задает график А.
    \item Пользователь задает мощность графика В.
    \item Пользователь задет график В.
  \end{enumerate}
  \item Выбор операции:
  \begin{enumerate}[label*=\arabic*.]
    \item Пользователь должен выбрать, какую из операций он хочет выполнить, в зависимости от его выбора будет выполнена операция из следующего списка:
    \begin{itemize}
      \item Объединение.
      \item Пересечение.
      \item Разность.
      \item Симметрическая разность.
      \item Композиция.
      \item Инверсия.
      \item Дополнение.
    \end{itemize}
    \item Если пользователь выбрал операцию объединения
    \begin{enumerate}[label*=\arabic*.]
      \item Переходим к пункту 3.
    \end{enumerate}
    \item Если пользователь выбрал операцию пересечения
    \begin{enumerate}[label*=\arabic*.]
      \item Переходим к пункту 4.
    \end{enumerate}
    \item Если пользователь выбрал операцию разности A и В
    \begin{enumerate}[label*=\arabic*.]
      \item Переходим к пункту 5.1
    \end{enumerate}
    \item Если пользователь выбрал операции разности В и А
    \begin{enumerate}[label*=\arabic*.]
      \item Переходим к пункту 5.2
    \end{enumerate}
    \item Если пользователь выбрал операцию симметрической разности
    \begin{enumerate}[label*=\arabic*.]
      \item Переходим к пункту 6.
    \end{enumerate}
    \item Если пользователь выбрал операцию дополнения A
    \begin{enumerate}[label*=\arabic*.]
      \item Переходим к пункту 7.1
    \end{enumerate}
    \item Если пользователь выбрал операцию дополнения В
    \begin{enumerate}[label*=\arabic*.]
      \item Переходим к пункту 7.2
    \end{enumerate}
    \item Если пользователь выбрал композицию графиков А и B 
    \begin{enumerate}[label*=\arabic*.]
      \item Переходим к пункту 8.1
    \end{enumerate}
    \item Если пользователь выбрал композицию графиков В и А
    \begin{enumerate}[label*=\arabic*.]
      \item Переходим к пункту 8.2
    \end{enumerate}
        \item Если пользователь выбрал инверсию графика В
    \begin{enumerate}[label*=\arabic*.]
      \item Переходим к пункту 9.1
    \end{enumerate}
        \item Если пользователь выбрал инверсию графика А
    \begin{enumerate}[label*=\arabic*.]
      \item Переходим к пункту 9.2
    \end{enumerate}
  \end{enumerate}
  \item \textbf{Операция объединения}:
  \begin{enumerate}[label*=\arabic*.]
    \item Создается пустой график С, который будет результатом операции.
    \item Выбираем первый компоненту из графика А.
    \item Записываем выбранный компоненту из графика А в график С.
    \item Если выбранная компонента графика А является последним:
    \begin{enumerate}[label*=\arabic*.]
      \item Переходим к пункту 3.6.
    \end{enumerate}
    \item Выбираем следующую компоненту графика А
    \begin{enumerate}[label*=\arabic*.]
      \item Переходим к пункту 3.3.
    \end{enumerate}
    \item Выбираем первую компоненту графика А.
    \item Выбираем первый компоненту графика В.
    \item Сравниваем выбранный компоненту из графика А с выбранной компонентой из графика В.
    \begin{enumerate}[label*=\arabic*.]
      \item Если выбранная компонента А не равна выбранной компоненте из графика В и выбранная компонента из графика А не является последним
      \begin{enumerate}[label*=\arabic*.]
        \item Выбираем следующую компоненту из графика А
        \begin{enumerate}[label*=\arabic*.]
          \item Переходим к пункту 3.8.
        \end{enumerate}
      \end{enumerate}
      \item Если выбранныая компоненту из графика А является последней и не равна выбранной элементу компоненте из графика В
      \begin{enumerate}[label*=\arabic*.]
        \item Переходим к пункту 3.10.
      \end{enumerate}
      \item Если выбранный компонента из графика А равена выбранной компоненте из графика В
      \begin{enumerate}[label*=\arabic*.]
        \item Переходим к пункту 3.9.
      \end{enumerate}
    \end{enumerate}
    \item Рассмотрим следующую компоненту из графика В.
    \begin{enumerate}[label*=\arabic*.]
      \item Переходим к пункту 3.8.
    \end{enumerate}
    \item Записываем выбранная компонента из графика В в график С.
    \begin{enumerate}[label*=\arabic*.]
      \item Если выбранная компонента из графика В является последней
      \begin{enumerate}[label*=\arabic*.]
        \item Переходим к пункту 3.12.
      \end{enumerate}
      \item Выбираем следующую компоненту из графика В и первую компоненту из графика А.
    \end{enumerate}
    \item Переходим к пункту 3.8.
    \item График С является графиком объединения графиков А и В.
    \item Алгоритм завершен.
  \end{enumerate}
  \item \textbf{Операция пересечения}
  \begin{enumerate}[label*=\arabic*.]
    \item Создаем пустой график D, который будет результатом операции.
    \item Выбираем первую компоненту из графикаый  А.
    \item Выбираем первую компоненту из графика В.
    \item Если выбранная компонента из графика А равна выбранной компоненте из графика В, то компонента графика В записывается в график D.
    \item Если выбранная компонента из графика В является последней.
    \begin{enumerate}[label*=\arabic*.]
      \item Переходим к пункту 4.7.
    \end{enumerate}
    \item Выбираем следующую компоненту из графика В.
    \begin{enumerate}[label*=\arabic*.]
      \item Переходим к пункту 4.4.
    \end{enumerate}
    \item Если выбранная компонента из графика А является последней.
    \begin{enumerate}[label*=\arabic*.]
      \item Переходим к пункту 4.9.
    \end{enumerate}
    \item Выбираем следующую компоненту из графика А.
    \begin{enumerate}[label*=\arabic*.]
      \item Переходим к пункту 4.4.
    \end{enumerate}
    \item График D является результатом пересечения графиков А и В.
    \item Алгоритм завершен.
  \end{enumerate}
  \item \textbf{Операция разности}
  \begin{enumerate}[label*=\arabic*.]
  \item \textbf{Операция разности A и B}
  \begin{enumerate}[label*=\arabic*.]
    \item Создадим пустой график D
    \item Возьмём первую компоненту из графика B.
    \item Возьмём первую компоненту из графика A.
    \item Если взятая компонента из графика B равна взятой компоненте из графика A.
    \begin{enumerate}[label*=\arabic*.]
      \item Переходим к пункту 5.1.9.
    \end{enumerate}
    \item Если взятая компонента из графика A является последней.
    \begin{enumerate}[label*=\arabic*.]
      \item Переходи к пункту 5.1.8.
    \end{enumerate}
    \item Если взятая компонента из графика A не является последней, возьмём следующую компоненту из графика A.
    \item Перейдём к пункту 5.1.4.
    \item Добавляем взятую компоненту из графика B в граФик D.
    \item Если взятая компонента из графика А является последней.
    \begin{enumerate}[label*=\arabic*.]
      \item Перейдём к пункту 10.
    \end{enumerate}
    \item Если взятая компонента из графика B не является последней, возьмём следующую компоненту из графика B.
    \item Перейдём к пункту 5.1.3.
  \end{enumerate}
  \item\textbf{Операция разности B и A}
  \begin{enumerate}[label*=\arabic*.]
    \item Создадим пустой график D.
    \item Возьмём первую компоненту из графика B.
    \item Возьмём первую компоненту из графика A.
    \item Если взятая компонента из графика B равен взятой компоненте из графика A.
    \begin{enumerate}[label*=\arabic*.]
      \item Переходим к пункту 5.2.9.
    \end{enumerate}
    \item Если взятая компонента из графика A является последней.
    \begin{enumerate}[label*=\arabic*.]
      \item Перейдём к пункту 5.2.8
    \end{enumerate}
    \item Если взятая компонента из графика A не является последней, возьмём следующую компоненту из графика A.
    \item Перейдём к пункту 5.2.4.
    \item Добавляем взятую компоненту из графика B в график D.
    \item Если взятая компонента из графика А является последней.
    \begin{enumerate}[label*=\arabic*.]
      \item Перейдём к пункту 10
    \end{enumerate}
    \item Если взятая компонента из графика B не является последней, возьмём следующую компоненту из графика B.
    \item Перейдём к пункту 5.2.3.
  \end{enumerate}
  \end{enumerate}
  \item \textbf{Симметрическая разность графиков А и В.}
  \begin{enumerate}[label*=\arabic*.]
    \item \textbf{Разность графиков А и В}
    \begin{enumerate}[label*=\arabic*.]
      \item Создадим пустой график C.
      \item Возьмём первую компоненту из графика А.
      \item Возьмём первую компоненту из графика В.
      \item Если взятая компонента из графика А равна взятой компоненте из графика В
      \begin{enumerate}[label*=\arabic*.]
        \item Переходим к пункту 6.1.9.
      \end{enumerate}
      \item Если взятая компонента из графика В является последней
      \begin{enumerate}[label*=\arabic*.]
        \item Перейдём к пункту 6.1.8
      \end{enumerate}
      \item Если взятая компонента из графика В не является последней, возьмём следующую компоненту из графика В.
      \item Перейдём к пункту 6.1.4.
      \item Добавляем взятую компоненту из графика А в график C.
      \item Если взятая компонента из графика А является последней.
      \begin{enumerate}[label*=\arabic*.]
        \item Перейдём к пункту 6.1.12.
      \end{enumerate}
      \item Если взятая компонента из графика А не является последней, возьмём следующую компоненту из графика А.
      \item Перейдём к пункту 6.1.3.
      \item C — разность графиков А и В.
    \end{enumerate}
    \item \textbf{Разность графиков В и А.}
    \begin{enumerate}[label*=\arabic*.]
      \item Создадим пусто1 график F.
      \item Возьмём первую компоненту из графика B.
      \item Возьмём первую компоненту из графика A.
      \item Если взятая компонентв из графика B равна взятой компоненте из графика A
      \begin{enumerate}[label*=\arabic*.]
        \item Переходим к пункту 6.2.9.
      \end{enumerate}
      \item Если взятая компонента из графика A является последней
      \begin{enumerate}
        \item Перейдём к пункту 6.2.8
      \end{enumerate}
      \item Если взятая компонента из графика A не является последней, возьмём следующую компоненту из графика A.
      \item Перейдём к пункту 6.2.4.
      \item Добавляем взятую компоненту из графика B в график F.
      \item Если взятая компонента из графика B является последней
      \begin{enumerate}[label*=\arabic*.]
        \item Перейдём к пункту 6.2.12.
      \end{enumerate}
      \item Если взятая компонента из графика B не является последней, возьмём следующую компоненту из графика B.
      \item Перейдём к пункту 6.2.3.
      \item F — разность графиков B и А.
    \end{enumerate}
    \item\textbf{Объединение графиков C и F.}
    \begin{enumerate}[label*=\arabic*.]
      \item Создаём новый пустой график D.
      \item Переписываем график С в график D.
      \item Возьмём первую компоненту из графика F.
      \item Возьмём первую компоненту из графика D.
      \item Если взятая компонента из графика F не равна взятой компоненте из графика D
      \begin{enumerate}[label*=\arabic*.]
        \item Переходим к пункту 6.3.7.
      \end{enumerate}
      \item Если взятая компонента из графика В равна  выбранной компоненте из графика D
      \begin{enumerate}[label*=\arabic*.]
        \item Переходим к пункту 6.3.11
      \end{enumerate}
      \item Если взятая компонента из графика D — последяя
      \begin{enumerate}[label*=\arabic*.]
        \item Переходим к пункту 6.3.10.
      \end{enumerate}
      \item Если взятая компонента из графика D — не последняя, то возьмём следующую компоненту из графика D.
      \item Перейдём к пункту 6.3.5.
      \item Добавляем взятую компоненту из графика F в график D.
      \item Если взятая компонента из графика F — последняя
      \begin{enumerate}[label*=\arabic*.]
        \item Переходим к пункту 10
      \end{enumerate}
      \item Если взятая компонента из графика F — не последняя, то возьмём следующую компоненту из графика F.
      \item Перейдём к пункту 6.3.4.
    \end{enumerate}
  \end{enumerate}
  \item \textbf{Операция дополнения}
  \begin{enumerate}[label*=\arabic*.]
    \item \textbf{Дополнение графика А}
    \begin{enumerate}[label*=\arabic*.]
      \item Зададим график U.
      \begin{enumerate}[label*=\arabic*.]
        \item Присвоим значение x равное 1.
        \item Если значение x больше 100, перейдём к пункту 7.1.2.
        \item Добавим значение x в график U.
        \item Увеличим значение x на 1.
        \item Перейдём к пункту 7.1.1.2.
      \end{enumerate}
      \item Разность графиков U и A.
      \begin{enumerate}[label*=\arabic*.]
        \item Создадим пустой график D.
        \item Возьмём первую компоненту из графика U.
        \item Возьмём первую компоненту из графика A.
        \item Если взятая компонента из графика U равна взятой компоненте из графика A, то переходим к пункту 7.1.2.11.
        \item Если взятая компонента из графика A является последней, то перейдём к пункту 7.1.2.10
        \item Если взятая компонента из графика A не является последней, возьмём следующую компоненту из графика A.
        \item Перейдём к пункту 7.1.2.4.
        \item Добавляем взятую компоненту из графика U в график D
        \item Если взятый элемент множества А является последним, перейдём к пункту 10.
        \item Если взятая компонента из графика U не является последней, возьмём следующую компоненту из графика U, то перейдём к пункту 7.1.2.3.
      \end{enumerate}
    \end{enumerate}
    \item \textbf{Дополнение графика B}
    \begin{enumerate}[label*=\arabic*.]
      \item Дополнение графика В
        \begin{enumerate}[label*=\arabic*.]
          \item Присвоим значение переменной x значение 1.
          \item Если значение x больше 100, перейдём к пункту 7.2.2
          \item Добавим значение x в график U.
          \item Увеличим значение x на 1.
          \item Перейдём к пункту 7.2.1.2
        \end{enumerate}
      \item Разность графиков U и B.
        \begin{enumerate}[label*=\arabic*.]
          \item Создадим пустой графика D.
          \item Возьмём первую компоненту из графика U.
          \item Возьмём первую компоненту из графика B.
          \item Если взятая компонента из графика U равна взятой компоненте из графика B, то переходим к пункту 7.2.2.8.
          \item Если взятая компонента из графика B является последней, то перейдём к пункту 7.2.2.7.
          \item Если взятая компонента из графика B не является последней, то возьмём следующую компоненту из графика B и перейдём к пункту 7.2.2.4.
          \item Добавляем взятыую компоненту из графика U в график D.
          \item Если взятая компонента из графика B является последней, то перейдём к пункту 10.
          \item Если взятая компонента из графика U не является последней, то возьмём следующую компоненту из графика U и перейдём к пункту 7.2.2.3.
        \end{enumerate}
      \end{enumerate}
    \item График D является результатом выполнения выбранной операции.
    \end{enumerate}
  \item \textbf{Операция композиции А и В}
    \begin{enumerate}[label*=\arabic*.]
    \item Создаем пустой график R
    \item Создаем переменную i, являющуюся номером элемента графика А, и присвоим ей значение 1
    \item Создаем переменную j, являющуюся номером элемента графика B, и присвоим ей значение 1
    \item Если j > m, то
        \begin{enumerate}[label*=\arabic*.]
            \item Увеличиваем значение i на единицу
            \item Выбираем первый элемент графика B
        \end{enumerate}
    \item Если i > n, то переходим к пункту 8.8.
    \item Если вторая компонента i-ого элемента равна первой компоненте j-ого элемента, то
        \begin{enumerate}[label*=\arabic*.]
            \item Создаём пару r
            \item Записываем первую компоненту элемента графика A на место первой компоненты r
            \item Записываем вторую компоненту элемента графика B на место второй компоненты r
            \item Добавляем пару r в график R
            \item Увеличиваем значение j на единицу
            \item Переходим к пункту 8.4.
        \end{enumerate}
    \item Если вторая компонента i-ого элемента не равна первой компоненте j-ого элемента, то
        \begin{enumerate}[label*=\arabic*.]
            \item Увеличиваем значение j на единицу
            \item Переходим к пункту 8.4.
        \end{enumerate}
    \item График R является графиком композиции графиков А и В
    \end{enumerate}
  \item \textbf{Операция композиции В и А}
    \begin{enumerate}[label*=\arabic*.]
        \item Создаем пустой график T
        \item Создаем переменную i, являющуюся номером элемента графика А, и присвоим ей значение 1
        \item Создаем переменную j, являющуюся номером элемента графика B, и присвоим ей значение 1
        \item Если i > n, то
        \begin{enumerate}[label*=\arabic*.]
            \item Увеличиваем значение j на единицу
            \item Выбираем первый элемент графика A
        \end{enumerate}
        \item Если j > m, то переходим к пункту 9.8.
        \item Если вторая компонента j-ого элемента равна первой компоненте i-ого элемента, то
        \begin{enumerate}[label*=\arabic*.]
            \item Создаём пару t.
            \item Записываем первую компоненту элемента графика B на место первой компоненты t.
            \item Записываем вторую компоненту элемента графика A на место второй компоненты t.
            \item Добавляем пару t в график Т.
            \item Увеличиваем значение i на единицу
            \item Переходим к пункту 9.4.
        \end{enumerate}
        \item Если вторая компонента j-ого элемента не равна первой компоненте i-ого элемента, то
        \begin{enumerate}[label*=\arabic*.]
            \item Увеличиваем значение i на единицу
            \item Переходим к пункту 9.4.
        \end{enumerate}
        \item График Т является графиком композиции графиков В и А
    \end{enumerate}
  \item \textbf{Операция инверсия графика А}
    \begin{enumerate}[label*=\arabic*.]
        \item Создаем пустой график X.
        \item Создаем переменную i, i - номер элемента графика А, и присваиваем ей значение 1
        \item Если i<n, то
        \begin{enumerate}[label*=\arabic*.]
            \item Создаём пару х
            \item Записываем первую компоненту элемента графика А на место второй компоненты х.
            \item Записываем вторую компоненту элемента графика А на место первой компоненты х
            \item Добавляем пару х в график Х.
            \item Увеличиваем значение i на единицу
            \item Переходим к пункту 10.3.
        \end{enumerate}
        \item График Х является графиком инверсии графика А
    \end{enumerate}
  \item \textbf{Операция инверсия графика В}
    \begin{enumerate}[label*=\arabic*.]
        \item Создаем пустой график Y.
        \item Создаем переменную j, j - номер элемента графика B, и присваиваем ей значение 1
        \item Если j<m, то
        \begin{enumerate}[label*=\arabic*.]
            \item Создаём пару у.
            \item Записываем первую компоненту элемента графика B на место второй компоненты y.
            \item Записываем вторую компоненту элемента графика B на место первой компоненты y.
            \item Добавляем пару у в график Y.
            \item Увеличиваем значение j на единицу
            \item Переходим к пункту 11.3
        \end{enumerate}
        \item График Y является графиком инверсии графика В
    \end{enumerate}
  \item Завершение алгоритма.
\end{enumerate}
\end{document}
